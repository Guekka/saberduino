\documentclass[12pt,twoside]{article}
% Français
\usepackage[french]{babel}
% Inclusion d'images
\usepackage{graphicx}
% Inclusion de PDF
\usepackage{pdfpages}

% Géométrie de la page
\usepackage[a4paper,width=150mm,top=25mm,bottom=25mm,bindingoffset=6mm]{geometry}
% En tête et bas de page
\usepackage{fancyhdr}
\pagestyle{fancy}
\fancyhead{}
\fancyhead[RO,LE]{Saberduino}
\fancyfoot{}
\fancyfoot[LE,RO]{\thepage}
\fancyfoot[CO,RE]{Bizel Edgar \& Phan Damien}
\renewcommand{\headrulewidth}{0.4pt}
\renewcommand{\footrulewidth}{0.4pt}

% Liens pour la table des matières
\usepackage{hyperref}
\hypersetup{colorlinks}

% Mode paysage
\usepackage{pdflscape}

\title{Saberduino}
\author{Bizel Edgar \& Phan Damien}
\date{3 décembre 2021}

\begin{document}
% Page de titre
\begin{titlepage}
	\begin{center}
		\vspace*{1cm}

		\Huge
		\textbf{Saberduino}

		\vspace{0.5cm}
		\LARGE
		Projet Arduino

		\vspace{1.5cm}

		\textbf{Bizel Edgar \& Phan Damien}

		\vfill

		\vspace{0.8cm}

		\includegraphics[width=0.6\textwidth]{logo_polytech.png}

		\vspace{0.8cm}

		\Large
		Électronique

		Polytech Nice Sophia

		3 décembre 2021

	\end{center}
\end{titlepage}


\tableofcontents

\vspace{0.5cm}

\section{Introduction}%
\label{sec:introduction}

\textbf{Saberduino} est un projet visant à créer un jeu vidéo sportif.

Lorsque nous avons dû choisir notre projet Arduino, nous avons rapidement décidé qu'il
s'agirait d'un jeu. Mais il aurait été dommage de se limiter à une simple console, alors
que de nombreuses possibilités s'ouvraient à nous.

Inspirés par le jeu de réalité virtuelle \href{https://store.steampowered.com/app/620980/Beat_Saber}{Beat
Saber}, nous avons ainsi décidé d'utiliser les mouvements du corps pour contrôler le jeu.

\section{Fonctionnalités}%
\label{sec:fonctionnalites}

\subsection{Principe du jeu}%
\label{sub:principe_du_jeu}

Un sabre à la main, nous devrons trancher les notes de musique qui arrivent à portée. La
tâche n'est pas simple : chaque bloc doit être tranché dans une direction spécifique.

Attention aux explosifs !

\subsection{Liste des fonctions}%
\label{sub:liste_des_fonctions}

Séparons la liste en deux. Un Arduino sera utilisé pour le sabre, et un autre pour
héberger le jeu.

Pour le sabre :
\begin{itemize}
	\item Communiquer par Bluetooth
	\item Signaler son inclinaison
	\item Signaler la direction du mouvement (gauche vers droite, haut vers bas\ldots)
	\item Vibration lors du choc avec un bloc
\end{itemize}
Pour la console :
\begin{itemize}
	\item Afficher le jeu : des cubes, marqués par une flèche indiquant la direction de
		tranchage
	\item Recevoir les informations du sabre et les traiter
	\item Émettre de la musique
\end{itemize}

\section{Matériel}%
\label{sec:materiel}

Le matériel sera initialement composé des éléments suivants :
\begin{itemize}
	\item Deux Arduinos. Si l'Arduino Uno s'avère insuffisant pour afficher notre jeu à
		une fréquence d'affichage correcte, nous nous tournerons vers
		une alternative telle que \href{https://www.pjrc.com/store/teensy40.html}{Teensy}.
	\item Un écran, de préférence en couleur. Sa définition sera comprise entre $96\times
	120$ et $240 \times 320$.
	\item Un émetteur et un récepteur bluetooth
	\item Des capteurs tels que des accéléromètres et des gyroscopes, pour la détection du
		mouvement
	\item Moteur à vibration
	\item Haut parleur
	\item Nous pensons pouvoir nous en passer, mais une caméra et un émetteur infrarouge
		pourraient être utiles afin de détecter la position du sabre.
\end{itemize}
\end{document}
